\documentclass[fleqn,reqno,12pt]{article}

%========================================
% Packages
%========================================

\usepackage[]{mfpackages}
\usepackage{mfenvironments}
\usepackage{mfcommands}

%========================================
% Bibliography
%========================================

\bibliography{references.bib}

%========================================
% General Layout Tweaks
%========================================

\usepackage[margin=2cm]{geometry}
% Itemize
\renewcommand{\labelitemi}{\large{$\mathbf{\cdot}$}}    % itemize symbols
\renewcommand{\labelitemii}{\large{$\mathbf{\cdot}$}}
\renewcommand{\labelitemiii}{\large{$\mathbf{\cdot}$}}
\renewcommand{\labelitemiv}{\large{$\mathbf{\cdot}$}}
% Description
\renewcommand{\descriptionlabel}[1]{\hspace\labelsep\textsc{#1}}

% Figure Captions
\usepackage{caption} % use corresponding myfiguresize!
\setlength{\captionmargin}{20pt}
\renewcommand{\captionfont}{\small}
\setlength{\belowcaptionskip}{7pt} % standard is 0pt

%========================================
% Define colors and comment functions
%========================================

\usepackage{xcolor}
\definecolor{firebrick}{RGB}{178,34,34}
\definecolor{DarkGreen}{RGB}{34,178,34} 
\definecolor{DarkOrange}{RGB}{255,100,50}
\newcommand{\mht}[1]{\textcolor{DarkOrange}{[mht: #1]}}  

%========================================
% Configuring the R code presentation
%========================================

\usepackage{listings}
\usepackage{color}
% the following defines the layout for the R code
\lstset{ %
  language=R,                     % the language of the code
  basicstyle=\footnotesize,       % the size of the fonts that are used for the code
  numbers=left,                   % where to put the line-numbers
  numberstyle=\tiny\color{gray},  % the style that is used for the line-numbers
  stepnumber=1,                   % the step between two line-numbers. If it's 1, each line
                                  % will be numbered
  numbersep=5pt,                  % how far the line-numbers are from the code
  backgroundcolor=\color{white},  % choose the background color. You must add \usepackage{color}
  showspaces=false,               % show spaces adding particular underscores
  showstringspaces=false,         % underline spaces within strings
  showtabs=false,                 % show tabs within strings adding particular underscores
  frame=single,                   % adds a frame around the code
  rulecolor=\color{black},        % if not set, the frame-color may be changed on line-breaks within not-black text (e.g. commens (green here))
  tabsize=2,                      % sets default tabsize to 2 spaces
  captionpos=b,                   % sets the caption-position to bottom
  breaklines=true,                % sets automatic line breaking
  breakatwhitespace=false,        % sets if automatic breaks should only happen at whitespace
  title=\lstname,                 % show the filename of files included with \lstinputlisting;
                                  % also try caption instead of title
  keywordstyle=\color{blue},      % keyword style
  commentstyle=\color{DarkGreen}, % comment style
  stringstyle=\color{DarkOrange}, % string literal style
  escapeinside={\%*}{*)},         % if you want to add a comment within your code
  morekeywords={*,...}            % if you want to add more keywords to the set
}

% this is for showing the R output
\lstnewenvironment{rc}[1][]{\lstset{language=R}}{}

% this is for inline R code
\newcommand{\ri}[1]{\lstinline{#1}}  %% Short for 'R inline'

%========================================
% Article Header 
%========================================


\title{Hands-on non-technical tutorial for Bayesian mixed effects regression \\ \normalsize{(using R and the \texttt{brms} package (with a focus on factorial designs))}}
\author{Michael Franke \& Timo Roettger}
\date{}

%========================================
% Article Body
%========================================

\begin{document}
\maketitle

\begin{abstract}
  Generalized linear models with mixed effects are very versatile and handy tools for statistical inference. Bayesian approaches to applying these models have also recently become more popular. This tutorial provides an accessible, non-technical introduction to the use and feel of Bayesian mixed effects regression models. The focus is on data from a factorial-design experiment. \\
  
  \medskip
  
  This tutorial should take you about 1 hour.
\end{abstract}

\tr{TRs START (suggestion): This tutorial is a very basic introduction to Bayesian inference using R (cite). We wrote this tutorial with a particular reader in mind. If you have used R before and if you have a basic understanding of linear regression, this tutorial is for you. If you don’t, you can easily catch up and do a crash course, for example with Bodo Winter’s tutorials here (link). When you think you are ready, this tutorial will walk you through a very simple example of bayesian inference for factorial designs. We tried to write this document as a practical departure point for people that want to move towards Bayesian inference using the wonderful brms package (Bürkner 2016). In comparison to other introduction, this tutorial remains very conceptual. We don’t discuss reasons for or against Bayesian inference and we don’t scare you away with mathematical details. After this tutorial, we want you to have a basic tool set to apply Bayesian inference to your data set. Concretely, you will spend an hour or so working your way through this document. Afterwards, you will be able to run a Bayesian analysis on your 2x2 factorial design and extract numbers that tell you something about your  data.

Now, what do we as experimental researchers often do? We often have a research hypothesis about how aspects of nature relate to each other. For example, we want to know whether voice pitch differs across females and males and whether it differs across social contexts and if so, we want to know how much, right?

To answer our questions, we lure a group of people into your lab, we ask them to say different words in different social contexts, we record their voices, and extract their pitch values. Our data might
look something like this}


Generalized linear regression models (GLMs) have become the new workhorse of statistical analyses. They are applicable in a wide variety of situation where otherwise a whole battery of statistical tests would be used, such as $t$-tests, ANOVA, binomial tests etc. But GLMs are more powerful than any one of these tests.  Unfortunately, they are also a tad difficult to understand, at least at first and when also considering hierarchical modeling (another way of saying ``mixed effects models''; more on this below). There is another thing that many people are talking about and many might be curious to learn more about. That is Bayesian approaches to statistical inference. Like GLMs, Bayesian data analysis has the (bad) reputation of being difficult (and obnoxiously subjective; whatever that may mean; again more below). This hands-on, ideology free and simple tutorial is here to the rescue. Based on a concrete standard use case, it tries to convey the very basics of Bayesian mixed effects models in a non-technical way.

\tr{Comment on intro MF: ANOVAs are basically GLMs so, I would not separate them. If you want to mention them maybe we want to say that they do the same thing.}

We have been inspired by a previous two-part tutorial by Bodo \citet{Winter2013:Linear-models-a} on mixed effects regression in a non-Bayesian ---a.k.a.~classical or frequentist--- paradigm. In a sense this could be considered part three of Bodo's nice and lofty introduction. But this tutorial is also readable without having read Bodo's stuff or, we hope, anything on either Bayesian data analysis or regression modeling. It's that basic we would like keep it. So no reason to be afraid! But also: no reason to be bored, because we \emph{will} cover all the essential concepts and we \emph{will} explain how to run and interpret the output of a Bayesian regression analysis using the excellent R package \texttt{brms} written by Paul CITE which makes Bayesian regression modeling 100\% accessible and ready-to-roll in the same way that another package has previously done for the non-Bayesian approaches (\texttt{lme4} CITE). 

\section{The data}

The tutorial by Bodo \citet{Winter2013:Linear-models-a} uses data from a paper on informal and polite speech in Korean \citep{WinterGrawunder2012:The-Phonetic-Pr}. Using R, let's first load the data by typing this into an R console (perhaps using RStudio \mf{insert URL})\footnote{If you are familiar with the previous tutorials by \citet{Winter2013:Linear-models-a}, it might help you to know that we have (cuteously) massaged the data a tiny little bit: we have already removed one line with missing data and we have changed the way the variable \texttt{scenario} is represented, so that it gets treated (correctly) as a factor later on without you having to manipulate it in some way or another.}

\begin{lstlisting}[language=R]
  # load the data into variable 'politeness_data'
  politeness_data = read.csv(file = 'MyData.csv')     
\end{lstlisting}

If we now type \ri{x <- rnorm(20)}, you will see the first six lines of the data, probably looking just like so:

\medskip


\begin{rc}
> head(politeness_data)
# A tibble: 6 x 5
  subject gender scenario attitude frequency
  <fct>   <fct>  <fct>    <fct>        <dbl>
1 F1      F      1        pol           213.
2 F1      F      1        inf           204.
3 F1      F      2        pol           285.
4 F1      F      2        inf           260.
5 F1      F      3        pol           204.
6 F1      F      3        inf           287.
\end{rc}


\printbibliography[heading=bibintoc]

\end{document}
